\documentclass[12pt,a4paper]{amsart}

\usepackage{fullpage}
\usepackage{macros}
\usepackage{bm}

% \usepackage[utf8]{inputenc}

\title{2-Wasserstein on a graph}

\author{Giovanni Pistone}
\address{de Castro Statistics, Collegio Carlo Alberto, Torino}

\date{\today}

\begin{document}

\maketitle

\section{Gradient and divergence on a Graph}

Let $G = (V,\omega)$ be a weighted simple connected graph, were
$\omega$ is a zero-diagonal non-negative symmetric matrix. The
adjacency matrix $E = [(\omega(x,y) > 0)]$ splits (in many ways) into
the sum of two adjacency matrices of directed graphs on $V$ in such a
way that $E = \overrightarrow E + \overleftarrow E$. Let $M(G)$ be the
vector space of matrices whose support is contained in the support of
$E$. Similarly, we define the space of symmetric matrices $S(G)$ and
the space of skew-symmetric matrices $A(G)$. Elements of $A(G)$ are
also called \emph{vector fields}. Both this special cases are uniquely
characterised by their restriction on either $\overrightarrow E$ or
$\overleftarrow E$.

For each given weight $\theta$ on $G$, that is, $\theta(x,y) \geq 0$
and $\theta(x,y) = 0$ if $\omega(x,y) = 0$, we define the \emph{inner
  product} $\scalarat \theta \cdot \cdot$ for $a,b \in M(G)$ by
\begin{multline*}
  \scalarat \theta a b = \frac 12 \sum _ {x,y \in V} a(x,y)
  b(x,y)\theta(x,y) = \frac 12 \sum _ {E(x,y)=1} a(x,y)
  b(x,y)\theta(x,y) = \\ \sum_{\overrightarrow E(x,y)=1}
  \frac12(a(x,y)b(x,y)+a(y,x)b(y,x)) \theta(x,y)\ .
\end{multline*}

If $a$ and $b$ are either both symmetric, or both skew-symmetric, then
\begin{equation*}
  \scalarat \theta a b = \sum_{\overrightarrow E(x,y)=1}
  a(x,y)b(x,y) \theta(x,y) \ .
\end{equation*}
If $a$ and $b$ are one symmetric and the other skew-symmetric, then
\begin{equation*}
  \scalarat \theta a b = 0 \ .
\end{equation*}

Let us call \emph{potential} a mapping $\Phi \colon V \to \reals$. We
can map potentials into vector fields by the operation of
\emph{$G$-gradient} $\nabla$ defined by,
\begin{equation*}
  \nabla \phi(x,y) = \sqrt {\omega(x,y)} (\phi(x) - \phi(y)) \ , \quad
  x,y \in V \ .
\end{equation*}
The meaning of the square rooth will be explained later. As the graph
$G$ is connected, it holds $\ker \nabla = \reals \one$.

Let us compute the adjoint of the gradient operator. For all vector
field $v \in A(G)$ it holds
\begin{multline*}
  \scalarat \theta {\nabla \phi}{b}  = \frac12 \sum_{E(x,y)=1}
  \sqrt{\omega(x,y)} \theta(x,y) (\phi(x) - \phi(y)) b(x,y) = \\
  = \frac12 \sum_{x \in V} \sum_{y \in N(x)} \sqrt{\omega(x,y)}
  \theta(x,y) \phi(x)
b(x,y) - \frac12 \sum_{x \in V} \sum_{y \in N(x)}
  \sqrt{\omega_{yx}}  \theta(y,x) \phi(x) b(y,x) = \\ \sum_{x \in V} \left(\sum_{y \in N(x)} \sqrt{\omega(x,y)}  \theta(x,y) b(x,y) \right) \phi(x) =
  \scalarof \phi {\Div(\theta b)} \ ,
\end{multline*}
where $N(x) = \setof{y}{E(x,y)=1}$ and $\Div \colon A(G) \to \reals^V$ is defined by
\begin{equation*}
  \Div w (x) = \sum_{y \in N(x)} \sqrt {\omega(x,y)} w(x,y) \ . 
\end{equation*}

It holds $\Div(\theta b) = 0$ if, and only if, $b$ is $\theta$-orthogonal to
all vector fields of the form  $\nabla  \phi$. 

\section{Laplacian and Hodge decomposition}
\label{sec:laplacian}

We define the following symmetric form on potentials:
\begin{equation*}
  \Scalarat \theta \phi \psi = \scalarat \theta {\nabla\phi}
  {\nabla\Psi} = \scalarof {L\phi} \Psi \ , 
\end{equation*}
where $L(\theta) \phi = \Div(\theta\nabla\phi)$ and $L(\theta)$ is the
\emph{Laplacian}.

The kernel of the gradient operator is charaterised by
\begin{equation*}
  0 = \Scalarat \theta \phi \phi = \frac 12 \sum_{xy \in E} \omega(x,y)
  (\phi(x) - \phi(y))^2 \theta(x,y) \ , 
\end{equation*}
that is, $\phi(x) = \phi(y)$ for all $x \in E$, that is, $\phi$ is
constant. It follows that the symmetric form is a scalar product above
on the space of sum-zero potentials, that is, on the affine space of
the probability simplex on $V$.

Let us study the image of the gradient operator in the space of vector
fields. The vector field $w$ belongs to the  orthogonal of the image of
$\nabla$ if, for all $\phi$,
\begin{equation*}
  0 = \scalarat \theta {\nabla \phi} w = \scalarof \phi {\Div(\theta
    w)} \ ,
\end{equation*}
that is, if, and only if, $\Div(\theta W) = 0$. The following
statement follow easily. It is sometimes called \emph{Hodge decomposition}.

\emph{Every vector field $v \in A(G)$ admits the $\theta$-orthogonal
  decomposition $v = \nabla \phi + w$, where $\phi$ is a potential
  defined up to a constant and $\Div(\theta w) = 0$}

Clearly, the decomposition depends on $\theta$

\section{Continuity equation}
\label{sec:continuity}

Let $\theta$ be a smooth mapping from the probability simplex on $V$ to the
cone of positive $S(G)$, $\theta \colon \rho \mapsto \theta(\rho) \in
S_+(G)$. Given a regular curve $[0,1] \ni t \mapsto \Phi(t)$ in the
space of zero-sum potentials, the ordinary differential equation
\begin{equation*}
  \derivby t \rho(t) +  \Div(\theta(\rho(t))\nabla \phi(t)) = 0 \ , \quad
  \rho(0) = \rho^0 \in \mathcal P_+(V) \ ,
\end{equation*}
has a local solution in $\mathcal P_+(V)$ because
\begin{equation*}
  \derivby t \sum_x \rho(x;t) = \scalarof {\dot \rho(t)}{\one} = 
 - \scalarof {\Div(\theta(\rho(t))\nabla \phi(t))}{1} = \scalarat
 {\theta(\rho)} {\nabla \Phi(t)}{\nabla 1}
 = 0 \ .
\end{equation*}

Given $\rho^0, \rho^1 \in \mathcal P_+(V)$, consider the value of the problem
\begin{align*}
 \inf &\int_0^1 \Scalarat {\theta(\rho(t))} {\phi(t)}{\phi(t)} \ dt
 \\
 &\derivby t \rho(t) +  \Div(\theta(\rho)\nabla \Phi(t)) = 0 \\
 &\rho(0) = \rho^0 \ , \quad \rho(1) = \rho^1.
\end{align*}

Let us write $L(\rho)\Phi = \Div(\theta(\rho)\nabla \Phi)$ and
consider that the conjugate of the quadratic form
$\Scalarat {\theta(\rho)} \phi \phi = \frac 12 \scalarof {L(\rho)\phi}
\psi$ is of the form
\begin{equation*}
  \frac 12 \scalarof {L(\rho)^\dagger \alpha} {\alpha} = \frac12 =
  \scalarof {L(\rho)^\dagger L(\rho) \phi} {L(\rho)(\phi)} = \frac12
  {L(\rho) \phi} \phi \ . 
\end{equation*}

Writing now $\Phi = L(\rho)^\dagger \dot \rho$ in the problem above,
we obtain the equivalent problem
\begin{align*}
 \inf &\int_0^1 \scalarof {L(\rho(t))^\dagger \dot \rho(t)}{\dot \rho(t)} \ dt
 \\
 &\rho(0) = \rho^0 \ , \quad \rho(1) = \rho^1 \ .
\end{align*}

In conclusion, the value of the problem is a squared Riemannian
distance for the metric whose tensor is $L(\rho)^\dagger$

\end{document}
